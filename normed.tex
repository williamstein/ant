\chapter{Normed Spaces and Tensor Products}
Much of this chapter is preparation for what we will do later
when we will prove that if~$K$ is complete with respect to a valuation
(and locally compact) and~$L$ is a finite extension of~$K$, then there
is a {\em unique} valuation on~$L$ that extends the valuation on~$K$.
Also, if~$K$ is a number field, $v=\absspc{}$ is a valuation on~$K$,
$K_v$ is the completion of~$K$ with respect to~$v$, and~$L$ is a
finite extension of~$K$, we'll prove that
$$
 K_v \tensor_K L   = \bigoplus_{j=1}^J L_j,
 $$
 where the $L_j$ are the completions of~$L$ with respect to the
 equivalence classes of extensions of~$v$ to~$L$.  In particular,
 if~$L$ is a number field defined by a root of $f(x)\in \Q[x]$, then
$$
  \Q_p  \tensor_\Q  L = \bigoplus_{j=1}^J L_j,
$$
 where the $L_j$ correspond to the irreducible factors of
 the polynomial $f(x) \in \Q_p[x]$ (hence the extensions of 
$\absspc_p$ correspond to irreducible factors of $f(x)$
over $\Q_p[x]$).  

In preparation for this clean view of the local nature of number
fields, we will prove that the norms on a finite-dimensional
vector space over a complete field are all equivalent.  We will also
explicitly construct tensor products of fields and deduce some of
their properties.

\section{Normed Spaces}
\begin{definition}[Norm]\label{defn:norm} 
Let $K$ be a field with valuation $\absspc$ and let $V$ be a vector space 
over $K$.  A real-valued function $\normspc$ on $V$ is called a \defn{norm} if 
\begin{enumerate}
\item $\norm{v}>0$ for all nonzero $v\in V$ (positivity).
\item $\norm{v+w} \leq \norm{v} + \norm{w}$ for all $v,w\in V$ (triangle inequality). 
\item $\norm{av} = \abs{a}\norm{v}$ for all $a\in K$ and $v\in V$ (homogeneity).
\end{enumerate} 
\end{definition}
Note that setting $\norm{v}=1$ for all $v\neq 0$ does {\em not} define
a norm unless the absolute value on $K$ is trivial, as $1=\norm{av} =
\abs{a}\norm{v}=\abs{a}$.  We assume for the rest of this section
that $\absspc{}$ is not trivial.

\begin{definition}[Equivalent]
  Two norms $\normspc_1$ and $\normspc_2$ on the same vector space~$V$
  are \defn{equivalent} if there exists positive real numbers $c_1$ and $c_2$
  such that for all $v\in V$
$$
  \norm{v}_1 \leq c_1 \norm{v}_2
  \qquad\text{and}\qquad
  \norm{v}_2 \leq c_2 \norm{v}_1.
$$
\end{definition}

\begin{exercise}{ex:normed1}
Suppose $\normspc_1$ and $\normspc_2$ are 
equivalent norms on a finite-dimensional vector space
$V$ over a field $K$ (with valuation $\absspc$). 
Carefully prove that the topology induced by $\normspc_1$
is the same as that induced by $\normspc_2$.
\end{exercise}


\begin{lemma}\label{lem:ext_unique}\ilem{any two norms equivalent}
Suppose that $K$ is a field that is complete with respect to a valuation
$\absspc{}$ and that $V$ is a finite dimensional~$K$ vector space.  
Then any two norms on $V$ are equivalent.
\end{lemma}
\begin{remark}
  As we shall see soon (see Theorem~\ref{thm:extensions}), the lemma
  is usually false if we do not assume that~$K$ is complete.  For
  example, when $K=\Q$ and $\absspc_p$ is the $p$-adic valuation, and
  $V$ is a number field, then there may be several extensions of
  $\absspc_p$ to inequivalent norms on $V$.
\end{remark}
If two norms are equivalent then the corresponding topologies on~$V$
are equal, since very open ball for $\normspc_1$ is contained in an
open ball for $\normspc_2$, and conversely. (The converse is also
true, since, as we will show, all norms on~$V$ are equivalent.)
\begin{proof} 
Let $v_1,\ldots, v_N$ be a basis for~$V$.  Define the max
norm $\normspc_0$ by 
$$
\norm{\sum_{n=1}^N a_n v_n}_0 = \max \left\{\abs{a_n} : n=1,\ldots, N\right\}.
$$
It is enough to show that any norm $\normspc$ is equivalent to
$\normspc_0$.  We have
\begin{align*}
\norm{\sum_{n=1}^N a_n v_n} & \leq
    \sum_{n=1}^N \abs{a_n} \norm{v_n} \\
    &\leq \sum_{n=1}^N \max{\abs{a_n}} \norm{v_n}\\
    & = c_1 \cdot \norm{\sum_{n=1}^N a_n v_n}_0,
\end{align*}
where $c_1 = \sum_{n=1}^N \norm{v_n}$.

To finish the proof, we show that there is a 
$c_2\in \R$ such that for all $v\in V$,
$$
 \norm{v}_0 \leq c_2 \cdot \norm{v}.
$$
We will only prove this in the case when $K$ is not just merely complete
with respect to $\absspc{}$ but also locally compact.  This will 
be the case of primary interest to us.  For a proof in the general case,
see the original article by Cassels (page 53). 

By what we have already shown, the function $\norm{v}$ is continuous
in the $\normspc_0$-topology, so by local compactness it attains its
lower bound $\delta$ on the unit circle $\left\{v\in V :
  \norm{v}_0=1\right\}$.  (Why is the unit circle compact?  With
respect to $\normspc_0$, the topology on $V$ is the same as that of a
product of copies of $K$.  If the valuation is archimedean then
$K\isom \R$ or $\C$ with the standard topology and the unit circle is
compact.  If the valuation is non-archimedean, then we saw (see
Remark~\ref{rem:locally_compact}) that if~$K$ is locally compact, then
the valuation is discrete, in which case we showed that the unit disc
is compact, hence the unit circle is also compact since it is closed.)
Note that $\delta>0$ by part 1 of Definition~\ref{defn:norm}.  Also,
by definition of $\normspc_0$, for any $v\in V$ there exists $a\in K$
such that $\norm{v}_0 = \abs{a}$ (just take the max coefficient in our
basis).  Thus we can write any $v\in V$ as $a\cdot w$ where $a\in K$
and $w\in V$ with $\norm{w}_0=1$.  We then have
$$
\frac{\norm{v}_0}{\norm{v}} = 
\frac{\norm{aw}_0}{\norm{aw}}
= \frac{\abs{a}\norm{w}_0}{\abs{a}\norm{w}}
= \frac{1}{\norm{w}} \leq \frac{1}{\delta}.
$$
Thus for 
all~$v$ we have
$$\norm{v}_0\leq c_2\cdot \norm{v},$$
where $c_2 = 1/\delta$, which proves the theorem. 
\end{proof}


\section{Tensor Products} \label{sec:tensor}
We need only a special case of the tensor product construction.
Let~$A$ and~$B$ be commutative rings containing a field~$K$ and suppose that~$B$ is of finite dimension~$N$ over~$K$, say, with basis 
$$
  1=w_1, w_2, \ldots, w_N.
$$
Then~$B$ is determined up to isomorphism as a ring over~$K$
by the multiplication table $(c_{i,j,n})$ defined by
$$
   w_i \cdot w_j = \sum_{n=1}^N c_{i,j,n} \cdot w_n.
$$
We define a new ring~$C$ containing~$K$ whose elements are 
the set of all expressions
$$
\sum_{n=1}^N a_n \ww_n
$$
where the $\ww_n$ have the same multiplication rule
$$
   \ww_i \cdot \ww_j = \sum_{n=1}^N c_{i,j,n} \cdot \ww_n
$$
as the $w_n$. 

There are injective ring homomorphisms
$$  
i:A\hra C, \qquad i(a) = a \ww_1  \qquad \text{(note that $\ww_1=1$)}
$$
and 
$$
j:B\hra C, \qquad j\left(\sum_{n=1}^N c_n w_n\right) = \sum_{n=1}^N c_n \ww_n.
\qquad\quad\,\,\,\,\mbox{}
$$
Moreover~$C$ is defined, up to isomorphism, by~$A$ and~$B$ and is
independent of the particular choice of basis $w_n$ of~$B$ (i.e., a
change of basis of $B$ induces a canonical isomorphism of the $C$
defined by the first basis to the $C$ defined by the second basis).
We write
$$
  C = A\tensor_K B
$$
since~$C$ is, in fact, a special case of the ring tensor product.

\begin{exercise}\label{ex:normed2}
Prove that the ring $C$ defined in Section~\ref{sec:tensor} really is the tensor
product of $A$ and $B$, i.e., that it satisfies the defining universal
mapping property for tensor products.  Part of this problem is for you
to look up a functorial definition of tensor product.
\end{exercise}



Let us now suppose, further, that~$A$ is a topological ring, i.e., has
a topology with respect to which addition and multiplication are
continuous.  Then the map 
$$
C\to A \oplus \cdots \oplus A,\qquad
  \sum_{m=1}^N a_m \ww_m \mapsto (a_1,\ldots, a_N)
$$
defines a bijection between~$C$ and the product of~$N$ copies of~$A$
(considered as sets). We give~$C$ the product topology.  It is readily
verified that this topology is independent of the choice of basis
$w_1, \ldots, w_N$ and that multiplication and addition on~$C$ are
continuous, so~$C$ is a topological ring.  We call this topology
on~$C$ the \defn{tensor product topology}.

Now drop our assumption that~$A$ and~$B$ have a topology, but suppose
that~$A$ and~$B$ are not merely rings but fields.  Recall that a
finite extension $L/K$ of fields is \defn{separable} if the number of
embeddings $L\hra \Kbar$ that fix~$K$ equals the degree of~$L$
over~$K$, where $\Kbar$ is an algebraic closure of~$K$.  The primitive
element theorem from Galois theory asserts that any such extension is
generated by a single element, i.e., $L=K(a)$ for some $a\in L$.
\begin{lemma}\label{lem:tensor_prod}\ilem{structure of tensor product of fields}
  Let~$A$ and~$B$ be fields containing the field~$K$ and suppose
  that~$B$ is a separable extension of finite degree $N=[B:K]$.  Then
  $C=A\tensor_K B$ is the direct sum of a finite number of fields
  $K_j$, each containing an isomorphic image of~$A$ and an isomorphic
  image of~$B$.
\end{lemma}
\begin{proof}
  By the primitive element theorem, we have $B=K(b)$, where~$b$ is a
  root of some separable irreducible polynomial $f(x)\in K[x]$ of
  degree~$N$.  Then $1,b,\ldots, b^{N-1}$ is a basis 
for~$B$ over~$K$, so
$$
  A\tensor_K B = A[\bb] \isom A[x]/(f(x))
$$
where $1,\bb,\bb^2,\ldots,\bb^{N-1}$ are 
linearly independent over~$A$ and $\bb$ satisfies 
$f(\bb)=0$.

Although the polynomial $f(x)$ is irreducible as an element
of $K[x]$, it need not be irreducible in $A[x]$.  Since~$A$
is a field,  we have a factorization 
$$
   f(x) = \prod_{j=1}^J g_j(x)
$$
where $g_j(x)\in A[x]$ is irreducible.  The $g_j(x)$ are
distinct because $f(x)$ is separable (i.e., has distinct
roots in any algebraic closure).   

For each~$j$, let $\bb_j\in \overline{A}$ be a root of $g_j(x)$, where
$\overline{A}$ is a fixed 
algebraic closure of the field~$A$.  Let $K_j = A(\bb_j)$.
Then the map
\begin{equation}\label{eqn:tensor}
  \vphi_j : A\tensor_K B \to K_j
\end{equation}
given by sending any polynomial $h(\bb)$ in $\bb$ (where $h\in A[x]$)
to $h(\bb_j)$ is a ring homomorphism, because the image
of~$\bb$ satisfies the polynomial $f(x)$, and $A\tensor_K B\isom A[x]/(f(x))$.

By the Chinese Remainder Theorem, the maps from (\ref{eqn:tensor})
combine to define a ring isomorphism
$$
 A\tensor_K B \isom A[x]/(f(x)) \isom \bigoplus_{j=1}^J A[x]/(g_j(x))
   \isom \bigoplus_{j=1}^J K_j.
$$

Each $K_j$ is of the form $A[x]/(g_j(x))$, so contains an isomorphic
image of $A$.  It thus remains to show that the ring 
homomorphisms
$$
  \lambda_j : B \xra{b\,\mapsto 1\tensor b} A\tensor_K B \xra{\vphi_j} K_j
$$
are injections.  Since $B$ and $K_j$ are both fields, $\lambda_j$
is either the $0$ map or injective.  However, $\lambda_j$ is
not the $0$ map since $\lambda_j(1)=1\in K_j$.  
\end{proof}
\begin{example}
  If $A$ and $B$ are finite extensions of $\Q$, then $A\tensor_\Q B$
  is an algebra of degree $[A:\Q]\cdot [B:\Q]$. For example, suppose
  $A$ is generated by a root of $x^2+1$ and $B$ is generated by a root
  of $x^3-2$.  We can view $A\tensor_\Q B$ as either $A[x]/(x^3-2)$ or
  $B[x]/(x^2+1)$.  The polynomial $x^2+1$ is irreducible over $\Q$,
  and if it factored over the cubic field $B$, then there would be a
  root of $x^2+1$ in $B$, i.e., the quadratic field $A=\Q(i)$ would be
  a subfield of the cubic field $B=\Q(\sqrt[3]{2})$, which is
  impossible.  Thus $x^2+1$ is irreducible over $B$, so $A\tensor_\Q B
  = A.B = \Q(i,\sqrt[3]{2})$ is a degree $6$ extension of $\Q$.
  Notice that $A.B$ contains a copy~$A$ and a copy of~$B$. By the
  primitive element theorem the composite field $A.B$ can be generated
  by the root of a single polynomial. For example, the minimal
  polynomial of $i+\sqrt[3]{2}$ is $x^6 + 3x^4 - 4x^3 + 3x^2 + 12x +
  5$, hence $\Q(i+\sqrt[3]{2})=A.B$.
\end{example}

\begin{example}
  The case $A\isom B$ is even more exciting.  For example, suppose
  $A=B=\Q(i)$. Using the Chinese Remainder Theorem we have that
$$
  \Q(i)\tensor_\Q \Q(i) \isom \Q(i)[x]/(x^2+1)
\isom \Q(i)[x]/((x-i)(x+i))
\isom \Q(i) \oplus \Q(i),
$$
since $(x-i)$ and $(x+i)$ are coprime.  The last isomorphism
sends $a + b x$, with $a,b\in\Q(i)$, to $(a+bi, a-bi)$.
Since $\Q(i)\oplus \Q(i)$ has zero divisors, the tensor
product $\Q(i)\tensor_\Q \Q(i)$ must also have zero divisors.
For example, $(1,0)$ and $(0,1)$ is a zero divisor pair
on the right hand side, and we can trace back to the elements
of the tensor product that they define.  First, by solving
the system
$$ a+bi=1\qquad \text{ and }\qquad a-bi=0$$
we see that
$(1,0)$ corresponds to $a=1/2$ and $b=-i/2$, i.e., to the element
$$\frac{1}{2}- \frac{i}{2} x\in \Q(i)[x]/(x^2+1).$$ 
This element in turn
corresponds to 
$$
\frac{1}{2}\tensor 1 - \frac{i}{2}\tensor i \in \Q(i)\tensor_\Q\Q(i).
$$
Similarly the other element $(0,1)$ corresponds to 
$$
 \frac{1}{2}\tensor 1 + \frac{i}{2}\tensor i \in \Q(i)\tensor_\Q\Q(i).
$$
As a double check, observe that 
\begin{align*}
\left(\frac{1}{2}\tensor 1 - \frac{i}{2}\tensor i\right)\cdot
 \left(\frac{1}{2}\tensor 1 + \frac{i}{2}\tensor i\right)
&= \frac{1}{4}\tensor 1 + \frac{i}{4}\tensor i - \frac{i}{4}\tensor i
    -\frac{i^2}{4}\tensor i^2\\
 &= \frac{1}{4}\tensor 1 - \frac{1}{4}\tensor 1 = 0 \in \Q(i)\tensor_\Q\Q(i).
\end{align*}
Clearing the denominator of $2$ and writing $1\tensor 1 = 1$, we have
$(1-i\tensor i)(1+i\tensor i) = 0$, so $i\tensor i$ is a root of the
polynomimal $x^2-1$, and $i\tensor i$ is not $\pm 1$, so $x^2-1$ has
more than $2$ roots.

In general, to understand $A\tensor_K B$ explicitly 
is the same as factoring either the defining polynomial of~$B$
over the field~$A$, or factoring the defining polynomial of~$A$ 
over~$B$.
\end{example}

\begin{exercise}\label{ex:normed3}
Find a zero divisor pair in $\Q(\sqrt{5})\tensor_\Q\Q(\sqrt{5})$.
\end{exercise}

\begin{exercise}\label{ex:normed4}
\begin{enumerate}
\item Is $\Q(\sqrt{5})\tensor_\Q\Q(\sqrt{-5})$ a field?
\item Is $\Q(\sqrt[4]{5})\tensor_\Q\Q(\sqrt[4]{-5})\tensor_\Q\Q(\sqrt{-1})$ a field?
\end{enumerate}
\end{exercise}

\begin{exercise}\label{ex:normed5}
  Suppose $\zeta_5$ denotes a primitive $5$th root of unity.  For
  any prime $p$, consider the tensor product $\Q_p \tensor_\Q
  \Q(\zeta_5) = K_1\oplus \cdots \oplus K_{n(p)}$.  Find a simple
  formula for the number $n(p)$ of fields appearing in the
  decomposition of the tensor product $\Q_p \tensor_\Q \Q(\zeta_5)$.
  To get full credit on this problem your formula must be correct, but
  you do {\em not} have to prove that it is correct.
\end{exercise}



\begin{corollary}\label{cor:fcp}\icor{tensor products and characteristic polynomials}
  Let $a\in B$ be any element and let $f(x)\in K[x]$ be the
  characteristic polynomials of $a$ over $K$ and let $g_j(x)\in A[x]$
  (for $1\leq j \leq J$) be the characteristic polynomials of the
  images of~$a$ under $B\to A\tensor_K B \to K_j$ over $A$,
  respectively.  Then
\begin{equation}\label{eqn:fcp}
  f(x) = \prod_{j=1}^J g_j(X).
\end{equation}
\end{corollary}
\begin{proof}
  We show that both sides of (\ref{eqn:fcp}) are the characteristic
  polynomial $T(x)$ of the image of $a$ in $A\tensor_K B$ over $A$.
  That $f(x)=T(x)$ follows at once by computing the characteristic
  polynomial in terms of a basis $\ww_1,\ldots, \ww_N$ of $A\tensor_K
  B$, where $w_1,\ldots, w_N$ is a basis for $B$ over $K$ (this is
  because the matrix of left multiplication by $b$ on $A \tensor_K B$
  is exactly the same as the matrix of left multiplication on~$B$, so
  the characteristic polynomial doesn't change).  To see that $T(X) =
  \prod g_j(X)$, compute the action of the image of~$a$ in $A\tensor_K
  B$ with respect to a basis of
\begin{equation}\label{eqn:decomp}
  A\tensor_K B \isom \bigoplus_{j=1}^J K_j
\end{equation}
composed of basis of the individual extensions $K_j$ of $A$.  The
  resulting matrix will be a block direct sum of submatrices, each of
  whose characteristic polynomials is one of the $g_j(X)$.  Taking
  the product gives the claimed identity (\ref{eqn:fcp}).
\end{proof}

\begin{exercise}\label{ex:normed6}
Suppose $K$ and $L$ are number fields (i.e., finite
extensions of $\Q$).  Is it possible for the tensor
product $K\tensor_\Q L$ to contain a nilpotent element? 
(A nonzero element $a$ in a ring $R$ is \defn{nilpotent} if 
there exists $n>1$ such that $a^n=0$.)
\end{exercise}

\begin{corollary}\icor{completion, norms, and traces}
\icor{norms, traces, and completions}
For $a\in B$ we have 
$$
 \Norm_{B/K}(a) = \prod_{j=1}^J \Norm_{K_j/A}(a),
$$
and 
$$
 \Tr_{B/K}(a) = \sum_{j=1}^J \Tr_{K_j/A}(a),
$$
\end{corollary}
\begin{proof}
  This follows from Corollary~\ref{cor:fcp}.  First, the norm is $\pm$
  the constant term of the characteristic polynomial, and the constant
  term of the product of polynomials is the product of the constant
  terms (and one sees that the sign matches up correctly).  Second,
  the trace is minus the second coefficient of the characteristic
  polynomial, and second coefficients add when one multiplies
  polynomials:
  $$
  (x^n + a_{n-1}x^{n-1} + \cdots ) \cdot (x^m + a_{m-1}x^{m-1} +
  \cdots ) = x^{n+m} + x^{n+m-1} (a_{m-1} + a_{n-1}) + \cdots.
  $$
  One could also see both the statements by considering a matrix of
  left multiplication by $a$ first with respect to the basis of
  $\ww_n$ and second with respect to the basis coming from the left
  side of (\ref{eqn:decomp}).

\end{proof}



%%% Local Variables: 
%%% mode: latex
%%% TeX-master: "ant"
%%% End: 
