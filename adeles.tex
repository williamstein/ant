including at least all $\lambda$ for which $Y_\lambda$ is not defined, \chapter{Global Fields and Adeles}
\section{Global Fields}\label{sec:global_fields}
\begin{definition}[Global Field]
  A \defn{global field} is a number field or a finite separable
  extension of $\F(t)$, where $\F$ is a finite field, and $t$ is
  transcendental over $\F$.
\end{definition}including at least all $\lambda$ for which $Y_\lambda$ is not defined, 

In this chapter, we will focus attention on number fields, and leave
the function field case to the reader.

The following lemma essentially says that the denominator of an
element of a global field is only ``nontrivial'' at a finite number of
valuations.
\begin{lemma}\label{lem:absbig}\ilem{valuations such that $\abs{a}>1$}
Let $a\in K$ be a nonzero element of a global field $K$.  Then
there are only finitely many inequivalent valuations $\absspc$
of $K$ for which
$$
  \abs{a} > 1.
$$
\end{lemma}
\begin{proof}
  If $K=\Q$ or $\F(t)$ then the lemma follows by Ostrowski's
  classification of all the valuations on~$K$ (see
Theorem~\ref{thm:ostrowski}). For example,
  when $a=\frac{n}{d}\in\Q$, with $n,d\in \Z$, then the valuations
  where we could have $\abs{a}>1$ are the archimedean one, or the
  $p$-adic valuations $\absspc_p$ for which $p\mid d$.

Suppose now that $K$ is a finite extension of $\Q$, so
$a$ satisfies a monic polynomial
$$
  a^n + c_{n-1} a^{n-1} + \cdots + c_0 = 0,
$$
for some $n$ and $c_0,\ldots, c_{n-1}\in\Q$.
If $\absspc$ is a non-archimedean valuation on $K$, we have
\begin{align*}
  \abs{a}^n &= \abs{-(c_{n-1} a^{n-1} + \cdots + c_0)} \\
      &\leq \max(1,\abs{a}^{n-1})\cdot \max(\abs{c_0},\ldots,\abs{c_{n-1}}).
\end{align*}
Dividing each side by $\abs{a}^{n-1}$, we have
that
$$
   \abs{a} \leq \max(\abs{c_0},\ldots,\abs{c_{n-1}}),
$$
so in all cases we have
\begin{equation}\label{eqn:maxabs}
   \abs{a} \leq \max(1, \abs{c_0},
   \ldots,\abs{c_{n-1}})^{1/(n-1)}.
\end{equation}
We know the lemma for~$\Q$, so there are only finitely many
valuations~$\absspc$ on~$\Q$ such that the right hand side of
(\ref{eqn:maxabs}) is bigger than~$1$.  Since each valuation of~$\Q$
has finitely many extensions to~$K$, and there are only finitely many
archimedean valuations, it follows that there are only finitely many
valuations on~$K$ such that $\abs{a}>1$.
\end{proof}

Any valuation on a global field is either archimedean, or discrete
non-archimedean with finite residue class field, since this is true of
$\Q$ and $\F(t)$ and is a property preserved by extending a valuation
to a finite extension of the base field.  Hence it makes sense to talk
of normalized valuations.  Recall that the normalized $p$-adic
valuation on $\Q$ is $\abs{x}_p = p^{-\ord_p(x)}$, and if~$v$ is a
valuation on a number field~$K$ equivalent to an extension of
$\absspc_p$, then the normalization of $v$ is the composite of the
sequence of maps
$$
  K\hra K_v \xra{\Norm} \Q_p \xra{\absspc_p} \R,
$$
where $K_v$ is the completion of $K$ at $v$.

\begin{example}
Let $K=\Q(\sqrt{2})$, and let $p=2$.  Because $\sqrt{2}\not\in\Q_2$, there is
exactly one extension of $\absspc_2$ to~$K$, and
it sends $a=1/\sqrt{2}$ to
$$
  \abs{\Norm_{\Q_2(\sqrt{2})/\Q_2}(1/\sqrt{2})}^{1/2}_2 = \sqrt{2}.
$$
Thus the normalized valuation of $a$ is $2$.

There are two extensions of $\absspc_7$ to $\Q(\sqrt{2})$,
since $\Q(\sqrt{2})\tensor_\Q \Q_7 \isom \Q_7 \oplus \Q_7$,
as $x^2-2 = (x-3)(x-4)\pmod{7}$.  The image of $\sqrt{2}$
under each embedding into $\Q_7$ is a unit in $\Z_7$, so
the normalized valuation of $a=1/\sqrt{2}$ is, in both
cases, equal to $1$.  More generally, for any valuation
of $K$ of characteristic an odd prime $p$, the
normalized valuation of $a$ is $1$.

Since $K=\Q(\sqrt{2})\hra \R$ in two ways, there are exactly
two normalized archimedean valuations on $K$, and
both of their values on $a$ equal $1/\sqrt{2}$.
Notice that the product of the absolute values of $a$
with respect to all normalized valuations is
$$
   2 \cdot \frac{1}{\sqrt{2}} \cdot \frac{1}{\sqrt{2}} \cdot 1
   \cdot 1 \cdot 1 \cdots  = 1.
$$
This ``product formula'' holds in much more generality, as
we will now see.
\end{example}

\begin{theorem}[Product Formula]\label{thm:product_formula}\ithm{product formula}
Let $a\in K$ be a nonzero element of a global field~$K$.
Let $\absspc_v$ run through the normalized valuations
of $K$.  Then $\abs{a}_v=1$ for almost all $v$, and
$$
\prod_{\text{\rm all }v} \abs{a}_v = 1\qquad{\text{\rm (the product
    formula).}}
$$
\end{theorem}
We will later give a more conceptual proof of this
using Haar measure (see Remark~\ref{rem:conceptual_prod}).
\begin{proof}
By Lemma~\ref{lem:absbig}, we have $\abs{a}_v\leq 1$
for almost all~$v$.  Likewise, $1/\abs{a}_v = \abs{1/a}_v\leq 1$
for almost all~$v$, so $\abs{a}_v = 1$ for almost all~$v$.

Let $w$ run through all normalized valuations of $\Q$ (or of $\F(t)$),
and write $v\mid w$ if the restriction of $v$ to $\Q$ is equivalent to $w$.
Then by Theorem~\ref{thm:normprod},
$$
 \prod_{v} \abs{a}_v = \prod_w \left(\prod_{v\mid w} \abs{a}_v\right)
     = \prod_w \abs{\Norm_{K/\Q}(a)}_w,
$$
so it suffices to prove the theorem for $K=\Q$.

By multiplicativity of valuations, if the theorem is true for $b$ and
$c$ then it is true for the product $b c$ and quotient $b/c$ (when
$c\neq 0$). The theorem is clearly true for $-1$, which has valuation
$1$ at all valuations.  Thus to prove the theorem for $\Q$ it suffices
to prove it when $a=p$ is a prime number.  Then we have
$\abs{p}_\infty = p$, $\abs{p}_p = 1/p$, and for primes $q\neq p$ that
$\abs{p}_q = 1$.  Thus
$$\prod_v \abs{p}_v = p \cdot \frac{1}{p} \cdot 1 \cdot 1 \cdot 1 \cdots = 1,$$
as claimed.
\end{proof}
\begin{exercise}\label{ex:adeles1}
  Prove that the product formula holds for $\F(t)$ similar to the
  proof we gave in class using Ostrowski's theorem for $\Q$.  You may
  use the analogue of Ostrowski's theorem for $\F(t)$, which you had
  on the previous homework assignment~\ref{ex:valuations2}.
  (Don't give a measure-theoretic proof.)
\end{exercise}

If $v$ is a valuation on a field $K$, recall that
we let $K_v$ denote the completion of $K$ with respect to $v$. Also when
$v$ is non-archimedean, let
$$
  \O_v = \O_{K,v} = \{x \in K_v : \abs{x} \leq 1\}
$$
be the ring of integers of the completion.

\begin{definition}[Almost All]
We say a condition holds for \defn{almost all} elements
of a set if it holds for all but finitely many elements.
\end{definition}

We will use the following lemma later (see Lemma~\ref{lem:adelext}) to
prove that formation of the adeles of a global field
is compatible with base change.
\begin{lemma} \label{lem:ints_adeles}\ilem{matching integers}
Let $\omega_1,\ldots, \omega_n$ be a basis for $L/K$,
where $L$ is a finite separable extension of the global field
$K$ of degree~$n$.
Then for almost all normalized non-archimedean valuations $v$ on $K$ we
have
\begin{equation}\label{eqn:sum_int}
   \omega_1 \O_{v} \oplus \cdots \oplus \omega_n \O_{v}
      = \O_{w_1} \oplus \cdots \oplus \O_{w_g}
      \subset K_v\tensor_K L,
\end{equation}
where $w_1,\ldots, w_g$ are the extensions of $v$
to $L$.   Here we have identified $a\in L$ with
its canonical image in $K_v\tensor_K L$, and the direct
sum on the left is the sum taken inside the tensor
product (so directness means that the intersections are
trivial).
\end{lemma}
\begin{proof}
  The proof proceeds in two steps.  First we deduce easily from
  Lemma~\ref{lem:absbig} that for almost all $v$ the left hand side
of (\ref{eqn:sum_int}) is
  contained in the right hand side.  Then we use a trick involving
  discriminants to show the opposite inclusion for all but finitely
  many primes.

  Since $\O_v\subset \O_{w_i}$ for all $i$, the left hand side of
  (\ref{eqn:sum_int}) is contained in the right hand side if
  $\abs{\omega_i}_{w_j}\leq 1$ for $1\leq i\leq n$ and $1\leq j\leq
  g$.  Thus by Lemma~\ref{lem:absbig}, for all but finitely many~$v$
  the left hand side of (\ref{eqn:sum_int}) is contained in the right
  hand side.  We have just eliminated the finitely many primes
  corresponding to ``denominators'' of some $\omega_i$, and now only
  consider~$v$ such that $\omega_1,\ldots, \omega_n \in \O_{w}$ for
  all $w\mid v$.

  For any elements $a_1,\ldots, a_n \in K_v\tensor_K L$, consider the
  discriminant
  $$
  D(a_1,\ldots, a_n) = \det(\Tr(a_i a_j)) \in K_v,
  $$
  where the trace is induced from the $L/K$ trace.
  Since each $\omega_i$ is in each $\O_w$, for $w\mid v$, the
  traces $\Tr(\omega_i \omega_j)$ lie in $\O_v$, so
  $$d = D(\omega_1,\ldots, \omega_n)\in \O_v.$$
  Also note that $d\in
  K$ since each $\omega_i$ is in $L$.  Now suppose that
  $$
  \alpha = \sum_{i=1}^n a_i \omega_i \in \O_{w_1} \oplus \cdots
  \oplus \O_{w_g},
  $$
  with $a_i \in K_v$.  Then by properties of determinants for any
  $m$ with $1\leq m\leq n$, we have
  \begin{equation}\label{eqn:discsquare}
  D(\omega_1,\ldots, \omega_{m-1}, \alpha, \omega_{m+1}, \ldots, \omega_n)
    = a_m^2 D(\omega_1,\ldots, \omega_n).
  \end{equation}
  The left hand side of (\ref{eqn:discsquare}) is in $\O_v$, so the
  right hand side is as well, i.e.,
  $$
  a_m^2 \cdot d \in \O_v, \qquad\text{(for }m=1,\ldots, n\text{)},
  $$
  where $d\in K$. Since $\omega_1,\ldots, \omega_n$ are a basis for
  $L$ over $K$ and the trace pairing is nondegenerate, we have $d\neq
  0$, so by Theorem~\ref{thm:product_formula} we have $\abs{d}_v=1$
  for all but finitely many~$v$.  Then for all but finitely many~$v$
  we have that $a_m^2\in \O_v$.  For these $v$, that $a_m^2\in\O_v$
  implies $a_m\in \O_v$ since $a_m\in K_v$, i.e., $\alpha$ is in the
  left hand side of (\ref{eqn:sum_int}).
\end{proof}
\begin{example}
Let $K=\Q$ and $L=\Q(\sqrt{2})$.  Let $\omega_1 = 1/3$ and $\omega_2 = 2\sqrt{2}$.  In the first stage of the above proof we would eliminate
$\absspc_3$ because $\omega_2$ is not integral at $3$.  The discriminant
is
$$
 d = D\left(\frac{1}{3}, 2\sqrt{2}\right)
   =\det \mtwo{\frac{2}{9}}{0}{0}{16} = \frac{32}{9}.
$$
As explained in the second part of the proof, as long as $v\neq 2, 3$,
we have equality of the left and right hand sides in (\ref{eqn:sum_int}).
\end{example}


\section{Restricted Topological Products}

In this section we describe a topological tool, which we need in order
to define adeles (see Definition~\ref{def:adele}).

\begin{definition}[Restricted Topological Products]
  Let $X_\lambda$, for $\lambda\in\Lambda$, be a family of topological
  spaces, and for almost all~$\lambda$ let $Y_{\lambda}\subset
  X_{\lambda}$ be an open subset of $X_{\lambda}$.  Consider the space
  $X$ whose elements are sequences $\bx = \{x_\lambda\}_{\lambda\in
    \Lambda}$, where $x_\lambda\in X_\lambda$ for every $\lambda$, and
  $x_\lambda\in Y_{\lambda}$ for almost all~$\lambda$.  We give $X$ a
  topology by taking as a basis of open sets the sets $\prod
  U_{\lambda}$, where $U_{\lambda}\subset X_{\lambda}$ is open for all
  $\lambda$, and $U_{\lambda} = Y_{\lambda}$ for almost all $\lambda$.
  We call~$X$ with this topology the \defn{restricted topological
    product} of the $X_{\lambda}$ with respect to the $Y_{\lambda}$.
\end{definition}


\begin{corollary}\label{lem:xs}\icor{topology on adeles}
  Let $S$ be a finite subset of $\Lambda$, including at least all $\lambda$ for which $Y_\lambda$ is not defined, and let $X_S$ be the set of
  $\bx\in X$ with $x_\lambda\in Y_\lambda$ for all $\lambda\not\in S$,
  i.e.,
  $$
  X_S = \prod_{\lambda \in S} X_{\lambda} \times
  \prod_{\lambda\not\in S} Y_{\lambda} \subset X.
  $$
  Then $X_S$ is an open subset of $X$, and the topology induced on
  $X_S$ as a subset of $X$ is the same as the product topology.
\end{corollary}

The restricted topological product depends on the totality of the
$Y_{\lambda}$, but not on the individual $Y_{\lambda}$:
\begin{lemma}\ilem{restricted product}
  Let $Y_{\lambda}'\subset X_{\lambda}$ be open subsets, and suppose
  that $Y_{\lambda} = Y_{\lambda}'$ for almost all~$\lambda$.  Then
  the restricted topological product of the $X_\lambda$ with respect
  to the $Y_{\lambda}'$ is canonically isomorphic to the restricted
  topological product with respect to the $Y_{\lambda}$.
\end{lemma}

\begin{lemma}\label{lem:res_compact}\ilem{local compactness of restricted product}
  Suppose that the $X_\lambda$ are locally compact and that the
  $Y_\lambda$ are compact.  Then the restricted topological
product $X$ of the $X_\lambda$ is locally compact.
\end{lemma}
\begin{proof}
  For any finite subset $S$ of $\Lambda$, the open subset $X_S\subset
  X$ is locally compact, because by Lemma~\ref{lem:xs} it is a product
  of finitely many locally compact sets with an infinite product of
  compact sets.  (Here we are using Tychonoff's theorem from topology,
  which asserts that an arbitrary product of compact topological
  spaces is compact (see Munkres's {\em Topology, a first course},
  chapter 5).) Since $X=\cup_{S} X_S$, and the $X_S$ are open in $X$,
  the result follows.
\end{proof}

The following measure will be extremely important in deducing
topological properties of the ideles, which will be used in
proving finiteness of class groups.  See, e.g., the
proof of Lemma~\ref{lem:bignorm}, which is a key input
to the proof of strong approximation (Theorem~\ref{thm:strong}).
\begin{definition}[Product Measure]\label{defn:prodmeasure}
  For all $\lambda\in\Lambda$, suppose $\mu_\lambda$ is a measure on
  $X_\lambda$ with $\mu_\lambda(Y_\lambda) = 1$ when $Y_\lambda$ is
  defined.  We define the \defn{product measure} $\mu$ on $X$ to be
  that for which a basis of measurable sets is $$\prod_\lambda
  M_\lambda$$
  where each $M_\lambda\subset X_\lambda$ has finite
  $\mu_\lambda$-measure and
  $M_\lambda=Y_\lambda$ for almost all $\lambda$, and where
  $$
  \mu\left(\prod_\lambda M_\lambda\right) = \prod_\lambda
  \mu_\lambda(M_\lambda).
  $$
\end{definition}

\section{The Adele Ring}
Let $K$ be a global field.  For each normalized valuation $\absspc_v$ of $K$,
let $K_v$ denote the completion of $K$.  If $\absspc_v$ is
non-archimedean, let $\O_v$ denote the ring of integers of $K_v$.
\renewcommand{\AA}{\mathbb{A}}

\begin{definition}[Adele Ring]\label{def:adele}
  The \defn{adele ring} $\AA_K$ of $K$ is the topological ring whose
  underlying topological space is the restricted topological product
  of the $K_v$ with respect to the $\O_v$, and where addition and
  multiplication are defined componentwise:
\begin{equation}\label{eqn:adelearith}
(\bx \by)_v = \bx_v \by_v \qquad
(\bx + \by)_v = \bx_v + \by_v\qquad
\text{for }\bx, \by\in\AA_K.
\end{equation}
\end{definition}
It is readily verified that (i) this definition makes sense, i.e., if
$\bx, \by\in \AA_K$, then $\bx\by$ and $\bx+\by$, whose components are
given by (\ref{eqn:adelearith}), are also in $\AA_K$, and (ii) that
addition and multiplication are continuous in the $\AA_K$-topology, so
$\AA_K$ is a topological ring, as asserted.
Also,
Lemma~\ref{lem:res_compact} implies that $\AA_K$ is locally compact
because the $K_v$ are locally compact
(Corollary~\ref{cor:locally_compact}), and the $\O_v$ are
compact (Theorem~\ref{thm:compact}).

There is a natural continuous ring inclusion
\begin{equation}\label{eqn:princ_inc}
K\hra \AA_K
\end{equation}
that sends $x\in K$ to the adele every one of whose components is $x$.
This is an adele because $x\in \O_v$ for almost all $v$, by
Lemma~\ref{lem:absbig}.  The map is injective because each map $K\to
K_v$ is an inclusion.

\begin{definition}[Principal Adeles]
  The image of (\ref{eqn:princ_inc}) is the ring of \defn{principal
    adeles}.
\end{definition}
It will cause no trouble to identify $K$ with the principal adeles, so
we shall speak of~$K$ as a subring of $\AA_K$.

Formation of the adeles is compatible with base change, in the
following sense.
\begin{lemma}\label{lem:adelext}\ilem{base extension of adeles}
  Suppose $L$ is a finite (separable) extension of the global field
  $K$.  Then
\begin{equation}\label{eqn:baseext}
  \AA_K \tensor_K L \isom \AA_L
\end{equation}
both algebraically and topologically.  Under this isomorphism,
  $$L\isom K\tensor_K L \subset \AA_K \tensor_K L$$ maps isomorphically onto
  $L\subset \AA_L$.
\end{lemma}
\begin{proof}
Let $\omega_1,\ldots, \omega_n$
be a basis for $L/K$ and let $v$ run through the normalized valuations
on~$K$.  The left hand side of (\ref{eqn:baseext}), with
the tensor product topology, is the restricted product of the
tensor products
$$
  K_v \tensor_K L \isom K_v \cdot\omega_1 \oplus \cdots \oplus K_v\cdot \omega_n
$$
with respect to the integers
\begin{equation}\label{eqn:intsum}
   \O_v\cdot \omega_1 \oplus \cdots \oplus \O_v\cdot \omega_n.
 \end{equation}
 (An element of the left hand side is a finite linear combination $\sum
\bx_i \tensor a_i$ of adeles $\bx_i \in \AA_K$ and coefficients $a_i
\in L$, and there is a natural isomorphism from the ring of such formal
sums to the restricted product of the $K_v\tensor_K L$.)

We proved before (Theorem~\ref{thm:extensions}) that
 $$
  K_v \tensor_K L \isom L_{w_1} \oplus \cdots \oplus L_{w_g},
  $$
  where $w_1,\ldots, w_g$ are the normalizations of the extensions
  of $v$ to $L$.  Furthermore, as we proved using discriminants (see
  Lemma~\ref{lem:ints_adeles}), the above identification identifies
  (\ref{eqn:intsum}) with
$$
 \O_{w_1} \oplus \cdots \oplus \O_{w_g},
$$
for almost all~$v$.
Thus the left hand side of (\ref{eqn:baseext}) is the restricted
product of the $L_{w_1} \oplus \cdots \oplus L_{w_g}$
with respect to the $\O_{w_1} \oplus \cdots \oplus \O_{w_g}$.
But this is canonically isomorphic to the restricted product
of all completions $L_w$ with respect to $\O_w$, which
is the right hand side of (\ref{eqn:baseext}).  This
establishes an isomorphism between the two sides of (\ref{eqn:baseext})
as topological spaces.  The map is also a ring homomorphism, so
the two sides are algebraically isomorphic, as claimed.
\end{proof}

\begin{corollary}\label{cor:addstruct}\icor{$\AA_K^+$ and base extension}
Let $\AA_K^+$ denote the topological group obtained from the
additive structure on $\AA_K$.  Suppose $L$ is a finite seperable
extension of $K$.
 Then
$$
  \AA_L^+ = \AA_K^+ \oplus \cdots \oplus \AA_K^+,
\qquad ([L:K] \text{ summands}).
$$
In this isomorphism the additive group $L^+\subset \AA_L^+$ of the
principal adeles is mapped isomorphically onto $K^+\oplus \cdots
\oplus K^+$.
\end{corollary}
\begin{proof}
For any nonzero $\omega \in L$, the subgroup $\omega\cdot \AA_K^+$
of $\AA_L^+$ is isomorphic as a topological group to $\AA_K^+$
(the isomorphism is multiplication by $1/\omega$).  By
Lemma~\ref{lem:adelext}, we have isomorphisms
$$
\AA_L^+ = \AA_K^+ \tensor_K L
   \isom \omega_1\cdot \AA_K^+ \oplus \cdots \oplus \omega_n \cdot \AA_K^+
   \isom \AA_K^+ \oplus \cdots \oplus \AA_K^+.
$$
If $a \in L$, write $a=\sum b_i \omega_i$, with $b_i \in K$.
Then $a$ maps via the above map to
$$x = (\omega_1\cdot \{b_1\},\ldots, \omega_n \cdot \{b_n\}),$$
where $\{b_i\}$ denotes the principal adele defined by $b_i$.
Under the final map, $x$ maps to the tuple
$$(b_1,\ldots, b_n) \in K\oplus \cdots \oplus K \subset
\AA_K^+ \oplus \cdots \oplus \AA_K^+.$$
The dimensions of $L$ and of $K\oplus \cdots \oplus K$ over
$K$ are the same, so
this proves the final claim of the corollary.
\end{proof}

\begin{theorem}\label{thm:adelequo}\ithm{compact quotient of adeles}
  The global field $K$ is discrete in $\AA_K$ and the quotient
  $\AA_K^+/K^+$ of additive groups is compact in the quotient
  topology.
\end{theorem}
At this point Cassels remarks
\begin{quote}``It is impossible to conceive of any other uniquely
defined topology on $K$.  This metamathematical reason is more
persuasive than the argument that follows!''
\end{quote}
\begin{proof}
Corollary~\ref{cor:addstruct}, with $K$ for $L$ and $\Q$ or
$\F(t)$ for $K$, shows that it is enough to verify
the theorem for $\Q$ or $\F(t)$, and we shall do it
here for $\Q$.

To show that $\Q^+$ is discrete in $\AA_\Q^+$ it is enough, because of
the group structure, to find an open set $U$ that contains $0 \in
\AA_\Q^+$, but which contains no other elements of $\Q^+$.  (If
$\alpha\in\Q^+$, then $U+\alpha$ is an open subset of $\AA_\Q^+$
whose intersection with $\Q^+$ is $\{\alpha\}$.)
We take for $U$ the set of $\bx=\{x_v\}_v \in \AA_\Q^+$ with
$$
 \abs{x_\infty}_\infty < 1
\qquad\text{and}\qquad \abs{x_p}_p \leq 1 \quad \text{(all $p$)},
$$
where $\absspc_p$ and $\absspc_\infty$ are respectively the
$p$-adic and the usual archimedean absolute values on~$\Q$.
If $b\in \Q\cap U$, then in the first place $b\in \Z$
because $\abs{b}_p\leq 1$ for all $p$, and then $b=0$
because $\abs{b}_\infty<1$.  This proves that $K^+$
is discrete in $\AA_\Q^+$.  (If we leave out one valuation,
as we will see later (Theorem~\ref{thm:strong}), this theorem is
false---what goes wrong with the proof just given?)

Next we prove that the quotient $\AA_\Q^+/\Q^+$ is compact.
Let $W\subset \AA_\Q^+$ consist of the $\bx=\{x_v\}_v\in \AA_\Q^+$
with
$$
  \abs{x_\infty}_\infty \leq \frac{1}{2}\qquad\text{and}\qquad
   \abs{x_p}_p \leq 1\qquad\text{for all primes $p$}.
$$
We show that every adele $\by=\{y_v\}_v$ is of the form
$$
  \by = a + \bx, \qquad a\in \Q, \quad \bx\in W,
$$
which will imply that the compact set $W$ maps surjectively
onto $\AA_\Q^+ / \Q^+$.
Fix an adele $\by=\{y_v\}\in\AA_\Q^+$.  Since $\by$
is an adele, for each prime $p$ we can find
a rational number
$$
  r_p = \frac{z_p}{p^{n_p}}
\qquad \text{with} \quad z_p \in \Z \quad \text{and} \quad n_p \in \Z_{\geq 0}
$$
such that
$$
  \abs{y_p - r_p}_p \leq 1,
$$
and
$$
  r_p = 0 \qquad \text{almost all $p$}.
$$
More precisely, for the finitely
many $p$ such that $$y_p = \sum_{n\geq -\abs{s}} a_np^n \not\in\Z_p,$$ choose
$r_p$ to be a rational number that is the value of an appropriate truncation
of the $p$-adic expansion of $y_p$, and
when $y_p\in \Z_p$ just choose $r_p = 0$.
Hence $r=\sum_{p} r_p\in\Q$ is well defined.
The $r_q$ for $q\neq p$ do not mess up the inequality
$\abs{y_p - r}_p \leq 1$ since the
valuation $\absspc_p$ is non-archimedean and the $r_q$ do not have any $p$ in
their denominator:
$$\abs{y_p  - r}_p
   = \abs{y_p - r_p - \sum_{q\neq p} r_q}_p
   \leq \max\left(\abs{y_p - r_p}_p, \abs{\sum_{q\neq p} r_q}_p\right)
   \leq \max(1,1) = 1.
$$
Now choose $s\in\Z$ such that
$$
  \abs{y_\infty - r-s} \leq \frac{1}{2}.
$$
Then $a=r+s$ and $\bx = \by - a$ do what is required,
since $\by - a = \by - r - s$ has the desired property
(since $s\in\Z$ and the $p$-adic valuations are
non-archimedean).

Hence the continuous map $W\to \AA_\Q^+/\Q^+$ induced by the quotient
map $\AA_\Q^+ \to \AA_\Q^+/\Q^+$ is surjective.  But $W$ is compact
(being the topological product of the compact spaces
$\abs{x_\infty}_\infty\leq 1/2$ and the $\Z_p$ for all $p$), hence
$\AA_\Q^+/\Q^+$ is also compact.
\end{proof}

\begin{exercise}\label{ex:adeles2}
  Prove Theorem~\ref{thm:adelequo}, that ``The global field $K$
  is discrete in $\AA_K$ and the quotient $\AA_K^+/K^+$ of additive
  groups is compact in the quotient topology.'' in the case when $K$
  is a finite extension of $\F(t)$, where $\F$ is a finite field.
\end{exercise}


\begin{corollary}\label{cor:subsetW}\icor{compact subset of adeles}
There is a subset $W$ of $\AA_K$ defined by inequalities of the
type $\abs{x_v}_v \leq \delta_v$, where $\delta_v=1$
for almost all $v$, such that every $\by\in\AA_K$ can
be put in the form
$$
  \by = a + \bx, \qquad a\in K, \quad \bx \in W,
$$
i.e., $\AA_K = K + W$.
\end{corollary}
\begin{proof}
  We constructed such a set for $K=\Q$ when proving
  Theorem~\ref{thm:adelequo}.  For general~$K$ the $W$ coming from the
  proof determines compenent-wise a subset of $\AA_K^+\isom
  \AA_\Q^+\oplus \cdots \oplus \AA_\Q^+$ that is a subset of a set
  with the properties claimed by the corollary.
\end{proof}

As already remarked, $\AA_K^+$ is a locally compact group, so it has
an invariant Haar measure.  In fact one choice of this Haar measure is
the product of the Haar measures on the $K_v$, in the sense
of Definition~\ref{defn:prodmeasure}.

\begin{corollary}\label{cor:finitemeasure}\icor{$\AA_K^+/K^+$ has finite measure}
The quotient $\AA_K^+/K^+$ has finite measure in the quotient measure
induced by the Haar measure on $\AA_K^+$.
\end{corollary}
\begin{remark}
This statement is independent of the particular choice
of the multiplicative constant in the Haar measure
on $\AA_K^+$.  We do not here go into the question of
finding the measure $\AA_K^+/K^+$ in terms of our
explicitly given Haar measure.  (See Tate's thesis,
\cite[Chapter XV]{cassels-frohlich}.)
\end{remark}
\begin{proof}
This can be reduced similarly to the case of $\Q$
or $\F(t)$ which is immediate, e.g., the $W$ defined
above has measure $1$ for our Haar measure.

Alternatively, finite measure follows from compactness.  To see
this,  cover
$\AA_K/K^+$ with the translates of $U$, where $U$ is a nonempty open
set with finite measure.  The existence of a finite subcover implies
finite measure.
\end{proof}

\begin{remark}\label{rem:conceptual_prod}
  We give an alternative proof of the product formula
  $\prod\abs{a}_v=1$ for nonzero $a\in K$.  We have seen that if
  $x_v\in K_v$, then multiplication by $x_v$ magnifies the Haar
  measure in $K_v^+$ by a factor of $\abs{x_v}_v$.  Hence if
  $\bx=\{x_v\}\in\AA_K$, then multiplication by $\bx$ magnifies the
  Haar measure in $\AA_K^+$ by $\prod \abs{x_v}_v$.  But now
  multiplication by $a\in K$ takes $K^+\subset \AA_K^+$ into $K^+$, so
  gives a well-defined bijection of $\AA_K^+/K^+$ onto $\AA_K^+/K^+$
  which magnifies the measure by the factor $\prod\abs{a}_v$.  Hence
  $\prod\abs{a}_v=1$ by Corollary~\ref{cor:finitemeasure}.  (The point is
  that if $\mu$ is the measure of $\AA_K^+/K^+$, then $\mu =
  \prod\abs{a}_v \cdot \mu$, so because $\mu$ is finite we must have
  $\prod\abs{a}_v = 1$.)
\end{remark}


\section{Strong Approximation}
We first prove a technical lemma and corollary, then use them to
deduce the strong approximation theorem, which is an extreme
generalization of the Chinese Remainder Theorem; it asserts that $K^+$
is dense in the analogue of the adeles with one valuation removed.

The proof of Lemma~\ref{lem:bignorm} below will use in a crucial way
the normalized Haar measure on $\AA_K$ and the induced measure on the
compact quotient $\AA_K^+/K^+$.  Since I am not formally developing
Haar measure on locally compact groups, and since I didn't explain
induced measures on quotients well in the last chapter, hopefully the
following discussion will help clarify what is going on.

The real numbers $\R^+$ under addition is a locally compact
topological group.  Normalized Haar measure $\mu$ has the property
that $\mu([a,b]) = b-a$, where $a\leq b$ are real numbers and
$[a,b]$ is the closed interval from $a$ to $b$.  The subset
$\Z^+$ of $\R^+$ is discrete, and the quotient $S^1 = \R^+/\Z^+$
is a compact topological group, which thus
has a Haar measure.  Let $\overline{\mu}$ be the Haar measure
on $S^1$ normalized so that  the natural quotient $\pi:\R^+\to S^1$
preserves the measure, in the sense that if $X\subset \R^+$
is a measurable set that maps injectively into $S^1$, then
$\mu(X) = \overline{\mu}(\pi(X))$.  This determine
$\overline{\mu}$ and we have $\overline{\mu}(S^1)=1$ since
$X=[0,1)$ is a measurable set that maps bijectively onto
$S^1$ and has measure~$1$.  The situation for the map
$\AA_K \to \AA_K/K^+$ is pretty much the same.


\begin{lemma}\label{lem:bignorm}
  There is a constant $C>0$ that depends only on the global field $K$
  with the following property:

Whenever $\bx=\{x_v\}_v \in \AA_K$ is such that
\begin{equation}\label{eqn:bignorm}
\prod_v \abs{x_v}_v > C,
\end{equation}
then there is a nonzero principal adele $a \in K\subset \AA_K$ such
that
$$
  \abs{a}_v \leq \abs{x_v}_v \qquad\text{\rm for all $v$}.
$$
\end{lemma}
\begin{proof}
This proof is modelled on Blichfeldt's proof of Minkowski's
Theorem in the Geometry of Numbers, and works in quite general
circumstances.

First we show that (\ref{eqn:bignorm}) implies that $\abs{x_v}_v=1$
for almost all $v$.  Because $\bx$ is an adele, we have
$\abs{x_v}_v\leq 1$ for almost all $v$.  If $\abs{x_v}_v<1$ for
infinitely many $v$, then the product in (\ref{eqn:bignorm}) would
have to be $0$.  (We prove this only when $K$ is a finite extension of
$\Q$.)  Excluding archimedean valuations, this is because the
normalized valuation $\abs{x_v}_v = \abs{\Norm(x_v)}_p$, which if less
than $1$ is necessarily $\leq 1/p$. Any infinite product of numbers
$1/p_i$ must be $0$, whenever $p_i$ is a sequence of primes.

Let $c_0$ be the Haar measure of $\AA_K^+/K^+$ induced from normalized
Haar measure on $\AA_K^+$, and let $c_1$ be the Haar measure of the set
of $\by=\{y_v\}_v \in \AA_K^+$ that satisfy
\begin{align*}
  \abs{y_v}_v &\leq \frac{1}{2} \qquad \text{if $v$ is real archimedean},\\
  \abs{y_v}_v &\leq \frac{1}{2} \qquad \text{if $v$ is complex archimedean},\\
  \abs{y_v}_v &\leq 1 \,\qquad \text{if $v$ is non-archimedean}.
\end{align*}
(As we will see, any positive real number $\leq 1/2$ would suffice in
the definition of $c_1$ above.  For example, in Cassels's article he
uses the mysterious $1/10$.  He also doesn't discuss the subtleties
of the complex archimedean case separately.)

Then $0<c_0<\infty$ since $\AA_K/K^+$ is compact, and $0<c_1<\infty$
because the number of archimedean valuations $v$ is finite.  We show
that $$C=\frac{c_0}{c_1}$$ will do.  Thus suppose $\bx$ is as in
(\ref{eqn:bignorm}).

The set $T$ of $\bt=\{t_v\}_v\in \AA_K^+$ such that
\begin{align*}
  \abs{t_v}_v &\leq \frac{1}{2}\abs{x_v}_v \,\qquad\text{if $v$ is real archimedean},\\
  \abs{t_v}_v &\leq \frac{1}{2}\sqrt{\abs{x_v}_v} \,\qquad\text{if $v$ is complex archimedean},\\
  \abs{t_v}_v &\leq \abs{x_v}_v \quad\,\qquad \text{if $v$ is non-archimedean}
\end{align*}
has measure
\begin{equation}\label{eqn:tbigger}
 c_1 \cdot \prod_{v} \abs{x_v}_v > c_1 \cdot C = c_0.
\end{equation}
(Note:  If there are complex valuations, then the some of
the $\abs{x_v}_v$'s in the product must be squared.)

%% (Note: The reason for the square root for the complex archimedean valuations
%% is that the normalized Haar measure on~$\C$ is the usual
%% Lebesgue measure on~$\C$, and the disc in $\C$ of radius
%% $(m_v/2)\cdot \sqrt{\abs{x_v}_v}$
%% has measure $(m_v\sqrt{\abs{x_v}_v})^2$ times the
%% measure ($=\pi/4$) of the disk of radius $1/2$ in $\C$.)

Because of (\ref{eqn:tbigger}), in
the quotient map $\AA_K^+ \to \AA_K^+/K^+$
there
must be a pair of distinct points of $T$ that have
the same image in $\AA_K^+/K^+$, say
$$
   \bt' = \{t'_v\}_v \in T\quad\text{and}\quad \bt'' = \{t''_v\}_v\in T
$$
and
$$
  a = \bt' - \bt'' \in K^+
$$
is nonzero.
Then
$$  \abs{a}_v = \abs{t'_v - t''_v}_v
         \leq
\begin{cases}
\abs{t'_v} + \abs{t''_v} \leq 2\cdot \frac{1}{2}\abs{x_v}_v \leq \abs{x_v}_v &
    \text{if $v$ is real archimedean, or}\\
\max(\abs{t'_v},\abs{t''_v}) \leq \abs{x_v}_v  &
    \text{if $v$ is non-archimedean,}
\end{cases}
$$
for all $v$.  In the case of complex archimedean $v$, we must be
careful because the normalized valuation $\absspc_v$ is the {\em
  square} of the usual archimedean complex valuation $\absspc_\infty$
on $\C$, so e.g., it does not satisfy the triangle inequality.
In particular, the quantity $\abs{t_v'-t''_v}_v$ is at most
the square of the maximum distance between two points in the disc in
$\C$ of radius $\frac{1}{2}\sqrt{\abs{x_v}_v}$, where by distance we
mean the usual distance.  This maximum distance in such a disc
is at most $\sqrt{\abs{x_v}_v}$, so $\abs{t_v'-t''_v}_v$ is at most
$\abs{x_v}_v$, as required.  Thus $a$ satisfies the requirements of
the lemma.
\end{proof}

\begin{corollary}\label{cor:small_a}
Let $v_0$ be a normalized valuation and let $\delta_v>0$ be given
for all $v\neq v_0$ with $\delta_v = 1$ for almost all $v$.  Then
there is a nonzero $a\in K$ with
$$
  \abs{a}_v \leq \delta_v\qquad\text{(all $v\neq v_0$)}.
$$
\end{corollary}
\begin{proof}
This is just a degenerate case of Lemma~\ref{lem:bignorm}.
Choose $x_v\in K_v$ with $0< \abs{x_v}_v \leq \delta_v$
and $\abs{x_v}_v=1$ if $\delta_v=1$.  We can then choose
$x_{v_0}\in K_{v_0}$ so that
$$
\prod_{\text{all $v$ including $v_0$}} \abs{x_v}_v > C.
$$
Then Lemma~\ref{lem:bignorm} does what is required.
\end{proof}

\begin{remark}
  The character group of the locally compact group $\AA_K^+$ is
  isomorphic to $\AA_K^+$ and $K^+$ plays a special role.  See Chapter
  XV of \cite{cassels-frohlich}, Lang's \cite{lang:algebraic_numbers},
  Weil's \cite{weil:adeles}, and Godement's Bourbaki seminars 171 and
  176.  This duality lies behind the functional equation of $\zeta$
  and $L$-functions.  Iwasawa has shown \cite{iwasawa:adele} that the
  rings of adeles are characterized by certain general
  topologico-algebraic properties.
\end{remark}

We proved before that $K$ is discrete in $\AA_K$.  If one valuation is
removed, the situation is much different.
\begin{theorem}[Strong Approximation]\label{thm:strong}\ithm{strong approximation}
  Let $v_0$ be any normalized nontrivial valuation of the global field~$K$.
  Let $\AA_{K,v_0}$ be the restricted topological product of the
  $K_v$ with respect to the $\O_v$, where $v$ runs through all
  normalized valuations $v\neq v_0$.  Then~$K$ is dense in
  $\AA_{K,v_0}$.
\end{theorem}
\begin{proof}
This proof was suggested by Prof. Kneser at the Cassels-Frohlich
conference.

Recall that if $\bx =\{x_v\}_v\in \AA_{K,v_0}$ then a basis of open
sets about $\bx$ is the collection of products
$$\prod_{v\in S} B(x_v,\eps_v) \times \prod_{v\not\in S,\,\, v\neq v_0} \O_v,$$
where $B(x_v,\eps_v)$ is an open ball in $K_v$ about $x_v$, and
$S$ runs through finite sets of normalized valuations (not including
$v_0$).  Thus
denseness of $K$ in $\AA_{K,v_0}$ is equivalent to the following
statement about elements.  Suppose we are given (i) a finite set $S$
of valuations $v\neq v_0$, (ii) elements $x_v\in K_v$ for all $v\in
S$, and (iii) an $\eps>0$.  Then there is an element $b\in K$ such that
$\abs{b-x_v}_v<\eps$ for all $v\in S$ and $\abs{b}_v\leq 1$ for all
$v\not\in S$ with $v\neq v_0$.

By the corollary to our proof that $\AA_K^+/K^+$ is compact
(Corollary~\ref{cor:subsetW}), there is a $W\subset \AA_K$ that is
defined by inequalities of the form $\abs{y_v}_v\leq \delta_v$ (where
$\delta_v=1$ for almost all $v$) such that ever $\mathbf{z}\in \AA_K$
is of the form
\begin{equation}\label{eqn:wsum}
  \mathbf{z} = \by + c, \qquad \by\in W, \quad c\in K.
\end{equation}
By Corollary~\ref{cor:small_a}, there is a nonzero $a\in K$ such
that
\begin{align*}
  \abs{a}_v &< \frac{1}{\delta_v}\cdot \eps\qquad \text{ for }v\in S,\\
  \abs{a}_v &\leq \frac{1}{\delta_v} \qquad\,\quad \text{ for } v\not\in S,\, v\neq v_0.
\end{align*}
Hence on putting $\mathbf{z} = \frac{1}{a}\cdot \bx$
in (\ref{eqn:wsum}) and multiplying by $a$, we see that
every $\bx\in \AA_K$ is of the shape
$$
  \bx = \mathbf{w} + b,\qquad \mathbf{w}\in a\cdot W, \quad b\in K,
$$
where $a\cdot W$ is the set of $a\by$ for $\by\in W$.
If now we let $\bx$ have components the given $x_v$ at $v\in S$,
and (say) $0$ elsewhere, then $b=\bx-\bw$ has the properties required.
\end{proof}

\begin{remark}
The proof gives a quantitative form of the theorem (i.e.,
with a bound for $\abs{b}_{v_0}$).  For an alternative approach,
see \cite{mahler:inequalities}.
\end{remark}

In the next chapter we'll introduce the ideles $\AA_K^*$.  Finally,
we'll relate ideles to ideals, and use everything so far to give a new
interpretation of class groups and their finiteness.

%%% Local Variables:
%%% mode: latex
%%% TeX-master: "ant"
%%% End:
