\chapter[Elliptic Curves and $L$-functions]{Elliptic Curves, Galois Representations, and $L$-functions}

This chapter is about elliptic curves and the central role they play
in algebraic number theory.  Our approach will be less systematic and
more a survey than most of the rest of this book.  The goal is to
give you a glimpse of the forefront of research by assuming many basic
facts that can be found in other books (see, e.g.,
\cite{silverman:aec}).

\section{Groups Attached to Elliptic Curves}


\begin{definition}[Elliptic Curve]\label{defn:ec}
  An \defn{elliptic curve} over a field~$K$ is a genus one curve~$E$
  defined over~$K$ equipped with a distinguished point $\O \in E(K)$.
  Here $E(K)$ is the set of all points on $E$ defined over $K$.
\end{definition}
We will not define \emph{genus} in this book, except to note that a
nonsingular curve over~$K$ has genus one if and only if over~$\Kbar$
it can be realized as a nonsingular plane cubic curve.\footnote{
	For a detailed and technical explanation of genus
	see \cite[Ch.~II.8]{hartshorne} or
	\cite[Ch.~7.3]{liu2006algebraic}
}
Moreover, one
can show (using the Riemann-Roch formula) that over any field a genus
one curve with a rational point can always be defined by a projective
cubic equation of the form
$$
  Y^2 Z + a_1 XYZ + a_3 YZ^2  = X^3  + a_2 X^2Z + a_4 XZ^2 + a_6 Z^3.
$$
In this form the distinguished point $\O$ is $(X:Y:Z) = (0:1:0)$.
Note that $\O$ is the only point on the curve with $Z=0$. So we
can consider the rest of the curve in the affine coordinates
by projecting onto the affine plane defined by $Z\neq 0$.
This gives the equation
\begin{equation}\label{weq}
  y^2 +a_1 xy + a_3 y = x^3 + a_2 x^2 + a_4 x + a_6.
\end{equation}
Thus one often presents an elliptic curve by giving a {\em Weierstrass
  equation} (\ref{weq}), though there are significant computational
advantages to other equations for curves (e.g., Edwards coordinates --
see work of Bernstein and Lange in \cite{bernstein2007inverted}).

Using \sage we plot an elliptic curve over the finite field
$\F_7$ and an elliptic curve curve defined over $\Q$.
\begin{sagecode} %skip
\begin{sagecell}
E = EllipticCurve(GF(7), [1,0])
E
\end{sagecell}
\begin{sageout}
Elliptic Curve defined by y^2 = x^3 + x over
    Finite Field of size 7
\end{sageout}
\begin{sagecell}
E.plot(pointsize=60, gridlines=True)
\end{sagecell}
\begin{sageout}[escapechar=!]
!\includegraphics[width=0.9\textwidth]{graphics/ecmod7}
\end{sageout}
\end{sagecode}

\begin{sagecode} %skip
\begin{sagecell}
E = EllipticCurve([1,0])
E
\end{sagecell}
\begin{sageout}
Elliptic Curve defined by y^2 = x^3 + x over
    Rational Field
\end{sageout}
\begin{sagecell}
E.plot()
\end{sagecell}
\begin{sageout}[escapechar=!]
!\includegraphics[width=0.9\textwidth]{graphics/ecq}
\end{sageout}
\end{sagecode}

Note that both plots above are of the affine equation $y^2 = x^3 + x$,
and do not include the distinguished point $\O$, which lies at
infinity.

\begin{remark}
The command {\tt{EllipticCurve}} in \Sage
can take as input a vector {\tt{[a4,a6]}}
of coefficients and returns the elliptic curve defined
by the Weirstrass equation seen in Equation~\ref{weq}
with $a_1=a_2=a_3=0$.
\end{remark}

\subsection{Abelian Groups Attached to Elliptic Curves}
If $E$ is an elliptic curve over~$K$, then we give the set
$E(K)$ of all $K$-rational points on~$E$ the structure of abelian
group with identity element~$\O$.  If we embed $E$ in the projective
plane, then this group is determined by the condition that three
points sum to the zero element $\O$ if and only if they lie on a
common line.
%TODO how to talk about tangent line. Expand section
%TODO to bring up Bezout's theorem (Silverman III.2 p 51)
%TODO or easy way to see use tangent line for point with itself?
%TODO avoid writing down formulas?
In our affine picture, a line will intersect the point at infinity
if it is vertical, or equivalently if it of the form $x=a$ for
some fixed $a\in K$.


\begin{example}\label{ex:ecgplaw}
On the curve $y^2=x^3-5x+4$, we have $(0,2) + (1,0) = (3,4)$.
This is because $(0,2)$, $(1,0)$, and $(3,-4)$ are on a common line
(given by the equation $y = 2 - 2x$) hence they sum to zero:
$$
  (0,2) + (1,0) + (3,-4) = \O.
$$
Notice $(3,4)$, $(3,-4)$, and $\O$ (the point at infinity on the curve) are also
on a common line (given by $x = 3$), so $(3,4)=-(3,-4)$.
We can illustration this in \sage:
\begin{sagecode}
\begin{sagecell}
E = EllipticCurve([-5,4])
E(0,2) + E(1,0)
\end{sagecell}
\begin{sageout}
(3 : 4 : 1)
\end{sageout}
\begin{sagecell} %skip

G = E.plot()
G += points ([(0,2) , (1,0) , (3,4) , (3,-4)],
    pointsize=90 , color='red', zorder=10)
G += line ([(-1,4) , (4,-6)] , color='black')
G += line ([(3,-6) , (3,6)] , color='black')
G.show()
\end{sagecell}
\begin{sageout}[escapechar=!] %skip
!\includegraphics[width=0.925\textwidth]{graphics/grouplaw}
\end{sageout}
\end{sagecode} %link

\noindent
Iterating the group operation often leads quickly to
very complicated points:

\begin{sagecode} %link
\begin{sagecell}
7*E(0,2)
\end{sagecell}
\begin{sageout}
(14100601873051200/48437552041038241 :
-17087004418706677845235922/10660394576906522772066289 :
 1)
\end{sageout}
\end{sagecode}
\end{example}

\begin{remark}
In the previous example we saw that iterating the
group operation led to points which used a lot of digits
to write down. This notion can be made formal and is called
the \emph{height} of the point. The height function is used
to prove the general Mordell-Weil theorem, see
\cite[Ch.~VIII.4]{silverman:aec}
\end{remark}

\begin{exercise}\label{ex:ec2torsion}
	Let $E$ be an elliptic curve given by a
	Weirstrass equation such as (\ref{weq}).
	Show that the points of order two are exactly
	the points on $E$ with $y$-coordinate equal to
	$0$.

	Hint: Recall that a point $P$ has order $2$ if
	$P + P + \O = \O$, which means the tangent line
	at $P$ goes through the point at infinity.
\end{exercise}

That the above condition---three points on a line sum to
zero---defines an abelian group structure on $E(K)$ is not obvious.
Depending on your perspective, the trickiest part is seeing that the
operation satisfies the associative axiom.  The best way to understand
the group operation on $E(K)$ is to view $E(K)$ as being related to a
class group.  As a first observation, note that the ring
$$
 R = K[x,y]/(y^2 +a_1 xy + a_3 y - (x^3 + a_2 x^2 + a_4 x + a_6))
$$
is a Dedekind domain, so $\Cl(R)$ is defined, and every nonzero
fractional ideal can be written uniquely in terms of prime ideals.
When $K$ is a perfect field, the prime ideals correspond to the Galois
orbits of affine points of $E(\overline{K})$.
Note that these do not include the point at infinity.

Let $\Div(E/K)$ be the free abelian group on the Galois orbits of
points of~$E(\overline{K})$, which as explained above is analogous to
the group of fractional ideals of a number field (here we {\em do}
include the point at infinity).
We call the elements of $\Div(E/K)$
{\em divisors}.  Let $\Pic(E/K)$ be the quotient of $\Div(E/K)$ by the
\emph{principal divisors}, i.e., the divisors associated to rational functions
$f\in K(E)^*$ via
$$
 f \mapsto (f) = \sum_{P} \ord_P(f) [P].
$$
Here $K(E)$ is the fraction field of the ring $R$ defined above.
Note that the principal divisor associated to $f$ is analogous to the
principal fractional ideal associated to a nonzero element of a number
field.  The definition of $\ord_P(f)$ is analogous to the ``power
of~$P$ that divides the principal ideal generated by~$f$''.
%TODO reference text for this? Hartshorne abstract non-singular curves
%TODO or somewhere in Silverman Ch VIII? or an algebra text on
%TODO valuations? or an exercise?
Define the \emph{class group} $\Pic(E/K)$ to be the quotient of the
divisors by the principal divisors, so we have
an exact sequence:
$$
  1\to K(E)^*/K^* \to \Div(E/K) \to \Pic(E/K) \to 0.
$$
%TODO is the 1 - ... - 0 weird?

A key difference between elliptic curves and algebraic number fields
is that the principal divisors in the context of elliptic curves all
have degree~$0$, i.e., the sum of the coefficients of the
divisor~$(f)$ is always~$0$.  This might be a familiar fact to you:
the number of zeros of a nonzero rational function on a projective
curve equals the number of poles, counted with multiplicity.  If we
let $\Div^0(E/K)$ denote the subgroup of divisors of degree~$0$, then
we have an exact sequence
$$
  1\to K(E)^*/K^* \to \Div^0(E/K) \to \Pic^0(E/K) \to 0.
$$

To connect this with the group law on $E(K)$, note that there
is a natural map
$$
 E(K) \to \Pic^0(E/K), \qquad P \mapsto [P-\O].
$$
Using the Riemann-Roch theorem, one can prove that this map
is a bijection, which is moreover an isomorphism of abelian groups.
Thus really when we discuss the group of $K$-rational
points on an $E$, we are talking
about the class group $\Pic^0(E/K)$.

Recall that we proved (Theorem~\ref{thm:finiteclassgrp}) that the
class group $\Cl(\O_K)$ of a number field is finite.
The  group $\Pic^0(E/K) =E(K)$ of an elliptic curve can be
either finite (e.g., for $y^2 + y = x^3 - x + 1$) or infinite (e.g.,
for $y^2 + y = x^3 - x$), and determining which is the case for any particular
curve is one of the central unsolved problems in number theory.

The Mordell-Weil theorem (see Chapter~\ref{ch:weakmw}) asserts that if $E$ is
an elliptic curve over a number field $K$, then there is a nonnegative integer
$r$, referred to as the \emph{algebraic rank of $E$}, such that
\begin{equation}\label{eqn:mw}
  E(\Q) \ncisom \Z^r \oplus T,
\end{equation}
where $T$ is a finite group.   This is similar to Dirichlet's unit theorem, which
gives the structure of the unit group of the ring of integers of a number field.
The main difference is that $T$ need not be cyclic, and computing $r$ appears to be
much more difficult than just finding the number of real and complex roots of
a polynomial!

\begin{example}
\sage has algorithms which can compute this rank for us.
For example we can compute the ranks of the curves
$y^2 + y = x^3 - x + 1$ and $y^2 + y = x^3 - x$ respectively.
\begin{sagecode}
\begin{sagecell}
EllipticCurve([0,0,1,-1,1]).rank()
\end{sagecell}
\begin{sageout}
0
\end{sageout}
\begin{sagecell}
EllipticCurve([0,0,1,-1,0]).rank()
\end{sagecell}
\begin{sageout}
1
\end{sageout}
\end{sagecode}
\end{example}

Also, if $L/K$ is an arbitrary extension of fields, and $E$ is an
elliptic curve over~$K$, then there is a natural inclusion
homomorphism $E(K)\hra E(L)$.  Thus instead of just obtaining one group
attached to an elliptic curve, we obtain a whole collection, one for
each extension of~$L$.  Even more generally, if $S/K$ is an arbitrary
scheme, then $E(S)$ is a group, and the association $S\mapsto E(S)$
defines a functor from the category of schemes to the category of
groups.  Thus each elliptic curve gives rise to map:
$$
 \left\{\text{Schemes over $K$}\right\} \longrightarrow
\left\{\text{Abelian Groups}\right\}
$$

\begin{remark}
	Elliptic curves are not the only objects that induce
	a functor from schemes to groups.
	\emph{Abelian varieties} are a larger class of
	schemes, which includes elliptic curves,
    that also induce such a functor.
    %%TODO: reference Milne
\end{remark}

\subsection{A Formula for Adding Points}

We close this section with an explicit formula for
adding two points in $E(K)$.
If $E$ is an elliptic curve over a field $K$,
given by an equation $y^2=x^3+ax+b$, then we
can compute the group addition using the following
algorithm.
\begin{algorithm}[Elliptic Curve Group Law]\label{alg:grouplaw}
Given $P_1, P_2\in E(K)$,
this algorithm computes the sum $R=P_1+P_2 \in E(K)$.
{\sf \begin{enumerate}
\item{}[One Point $\O$] If $P_1=\O$ set $R=P_2$ or if $P_2=\O$ set $R=P_1$
and terminate.  Otherwise write $P_i=(x_i,y_i)$.
\item{}[Negatives]  If $x_1 = x_2$ and $y_1 = -y_2$, set $R=\O$ and terminate.
\item{}[Compute $\lambda$]\label{alg:grouplaw_3}
Set $\ds \lambda = \begin{cases}
 (3x_1^2+a)/(2y_1) & \text{if }P_1 = P_2,\\
(y_1-y_2)/(x_1-x_2) & \text{otherwise.}
\end{cases}$\\
Note: If $y_1=0$ and $P_1=P_2$, output $\O$ and terminate.
\item{}[Compute Sum]\label{alg:grouplaw_4}  Then
$R = \ds \left(\lambda^2 -x_1 - x_2, -\lambda x_3 - \nu\right)$,
where $\nu = y_1 - \lambda x_1$ and~$x_3$ is the~$x$ coordinate of $R$.
\end{enumerate}}
\end{algorithm}

\subsection{Other Groups}
There are other abelian groups attached to elliptic curves, such as
the torsion subgroup $E(K)_{\tor}$ of elements of $E(K)$ of finite
order.  The torsion subgroup is (isomorphic to) the group $T$ that
appeared in Equation~\eqref{eqn:mw} above).  When $K$ is a number
field, there is a group called the Shafarevich-Tate group $\Sha(E/K)$
attached to~$E$, which plays a role similar to that of the class group
of a number field (though it is an open problem to prove that
$\Sha(E/K)$ is finite in general).  The  definition of $\Sha(E/K)$ involves Galois
cohomology, so we wait until Chapter~\ref{ch:gc} to define it.  There
are also component groups attached to~$E$, one for each prime of
$\O_K$.  These groups all come together in the Birch and
Swinnerton-Dyer conjecture (see \url{http://wstein.org/books/bsd/}).
%TODO change this reference please


\section{Galois Representations Attached to Elliptic Curves}
Let~$E$ be an elliptic curve over a number field~$K$.
In this section we attach representations of
$G_K = \Gal(\Kbar/K)$ to~$E$, and use them to define an $L$-function
$L(E,s)$.   This $L$-function is yet another generalization of the
Riemann Zeta function, that is different from the $L$-functions
attached to complex representations $\Gal(\Qbar/\Q)\to \GL_n(\C)$,
which we encountered before in Section~\ref{sec:artin}.

Fix an integer~$n$.  The group structure on~$E$ is defined by
algebraic formulas with coefficients that are elements of~$K$, so the subgroup
$$
 E[n] = \{P \in E(\Kbar) : nP = \O\}
$$
is invariant under the action of $G_K$ (but the action may act non-trivially on individual elements!).  We thus obtain a homomorphism
$$
  \rhobar_{E,n} : G_K \to \Aut(E[n]).
$$


For any positive integer~$n$, the group $E[n]$ is isomorphic as an
abstract abelian group to $(\Z/n\Z)^2$.  There are various
related ways to see why this is true. One is to use the Weierstrass
$\wp$-theory to parametrize $E(\C)$ by the the complex numbers, i.e.,
to find an isomorphism $\C/\Lambda \isom E(\C)$, where $\Lambda$ is a
lattice in $\C$ and the isomorphism is given by $z\mapsto
(\wp(z),\wp'(z))$ with respect to an appropriate choice of coordinates
on $E(\C)$.  It is then an easy exercise to verify that
$(\C/\Lambda)[n]\isom (\Z/n\Z)^2$.
For a detailed and rigorous walk through of this method see
\cite[Ch.~1.4]{diamond-shurman}.

Another way to understand $E[n]$ is to use that $E(\C)_{\tor}$ is isomorphic
%TODO do you mean \Qbar here?
to the quotient
$$\H_1(E(\C),\Q)/\H_1(E(\C),\Z)$$
of homology groups and that the homology of a curve
of genus~$g$ is isomorphic to $\Z^{2g}$.  Then
$$
 E[n]\isom (\Q/\Z)^2[n] = (\Z/n\Z)^2.
$$

\begin{example}
Consider the case when $n=2$. From Exercise~\ref{ex:ec2torsion}
we know that the points in $E[2]$ are exactly the points with
$y$-coordinate $0$. Let $E$ be the elliptic curve given by
$E: y^2 = x^3 + x + 1$. If $y=0$ then $x$ has to be a root
of the polynomial $x^3 + x + 1$, so the points in $E[2]$
are defined over the splitting field of $x^3 + x + 1$.
We can compute these points in \sage.

\begin{sagecode}
\begin{sagecell}
E = EllipticCurve([1,1]); E
\end{sagecell}
\begin{sageout}
Elliptic Curve defined by y^2 = x^3 + x + 1 over
    Rational Field
\end{sageout}
\begin{sagecell}
R.<x> = QQ[]; R
\end{sagecell}
\begin{sageout}
Univariate Polynomial Ring in x over Rational Field
\end{sageout}
\end{sagecode} %link
\begin{sagecode} %link
\begin{sagecell}
f = x^3 + x + 1
K.<a> = NumberField(f)
M.<b> = K.galois_closure(); M
\end{sagecell}
\begin{sageout}
Number Field in b with defining polynomial
    x^6 + 6*x^4 + 9*x^2 + 31
\end{sageout}
\end{sagecode} %link
\begin{sagecode} %link
\begin{sagecell}
F = E.change_ring(M)
T = F.torsion_subgroup(); T
\end{sagecell}
\begin{sageout}
Torsion Subgroup isomorphic to Z/2 + Z/2 associated
    to the Elliptic Curve defined by y^2 = x^3 + x + 1
    over Number Field in b with defining polynomial
    x^6 + 6*x^4 + 9*x^2 + 31
\end{sageout}
\end{sagecode} %link
\begin{sagecode} %link
\begin{sagecell}
T.gens()
\end{sagecell}
\begin{sageout}
((1/18*b^4 + 5/18*b^2 + 1/2*b + 2/9 : 0 : 1),
    (1/18*b^4 + 5/18*b^2 - 1/2*b + 2/9 : 0 : 1))
\end{sageout}
\end{sagecode}
\noindent
Note that this matches with what we expected: we computed
two generators for $E[2]$ corresponding to two generators of
$\left(\ZZ/2\ZZ\right)^2$.

\end{example}

If $n=p$ is a prime, then upon chosing a basis for the two-dimensional
$\F_p$-vector space $E[p]$, we obtain an isomorphism $\Aut(E[p]) \isom
\GL_2(\F_p)$.  We thus obtain a mod~$p$ Galois representation
$$
 \rhobar_{E,p} : G_K \to \GL_2(\F_p).
$$
This representation $\rhobar_{E,p}$ is continuous if $\GL_2(\F_p)$ is endowed with the
discrete topology, because the field
$$
 K(E[p]) = K(\{a,b : (a,b) \in E[p]\})
$$
is a Galois extension of~$K$ of finite degree.
%TODO is this hard to prove? is it an easy exercise?
%TODO can you do it with division polynomials similar
%TODO to the previous example somehow?

In order to attach an $L$-function to $E$, one could try to embed
$\GL_2(\F_p)$ into $\GL_2(\C)$ and use the construction of Artin
$L$-functions from Section~\ref{sec:artin}.
Unfortunately, this approach is doomed in general, since
$\GL_2(\F_p)$ frequently does not embed in $\GL_2(\C)$.
The following Sage session shows that for $p=5,7$, there are
no 2-dimensional irreducible representations of $\GL_2(\F_p)$,
so $\GL_2(\F_p)$ does not embed in $\GL_2(\C)$.
(The notation in the output below is {\tt [degree of rep, number of times it occurs]}.)
\begin{sagecode}
\begin{sagecell} %TODO this doesn't work this way
gap(GL(2,GF(2))).CharacterTable().CharacterDegrees()
\end{sagecell}
\begin{sageout}
[ [ 1, 2 ], [ 2, 1 ] ]
\end{sageout}
\begin{sagecell} %TODO this doesn't work this way
gap(GL(2,GF(3))).CharacterTable().CharacterDegrees()
\end{sagecell}
\begin{sageout}
[ [ 1, 2 ], [ 2, 3 ], [ 3, 2 ], [ 4, 1 ] ]
\end{sageout}
\begin{sagecell} %TODO this doesn't work this way
gap(GL(2,GF(5))).CharacterTable().CharacterDegrees()
\end{sagecell}
\begin{sageout}
[ [ 1, 4 ], [ 4, 10 ], [ 5, 4 ], [ 6, 6 ] ]
\end{sageout}
\begin{sagecell} %TODO this doesn't work this way
gap(GL(2,GF(7))).CharacterTable().CharacterDegrees()
\end{sagecell}
\begin{sageout}
[ [ 1, 6 ], [ 6, 21 ], [ 7, 6 ], [ 8, 15 ] ]
\end{sageout}
\end{sagecode}

Instead of using the complex numbers, we use the $p$-adic numbers, as
follows.  For each power $p^m$ of $p$, we have defined a homomorphism
$$
  \rhobar_{E,p^m}: G_K \to \Aut(E[p^m]) \ncisom \GL_2(\Z/p^m\Z).
$$
We combine together all of these representations (for all $m\geq 1$)
using the inverse limit.
Recall that the $p$-adic numbers are
$$
  \Z_p = \varprojlim \Z/p^m\Z,
$$
which is the set of all compatible choices of integers modulo $p^m$ for
all $m$.
We obtain a (continuous) homomorphism
$$
  \rho_{E,p}: G_K \to \Aut(\varprojlim E[p^m]) \isom \GL_2(\Z_p),
$$
where $\Z_p$ is the ring of $p$-adic integers.  The composition of
this homomorphism with the reduction map $\GL_2(\Z_p) \to \GL_2(\F_p)$
is the representation $\rhobar_{E,p}$, which we defined above, which
is why we denoted it by $\rhobar_{E,p}$. We
next try to mimic the construction of $L(\rho,s)$ from
Section~\ref{sec:artin} in the context of a $p$-adic Galois
representation $\rho_{E,p}$.

\begin{definition}[Tate module]
The \emph{$p$-adic Tate module of $E$} is
$$
  T_p(E) = \varprojlim E[p^n].
$$
\end{definition}

Let $M$ be the fixed field of $\ker(\rho_{E,p})$. The image of
$\rho_{E,p}$ is infinite, so $M$ is an infinite extension of~$K$.
Fortunately, one can prove that~$M$ is ramified at only finitely many
primes (the primes of bad reduction for $E$ and $p$---see \cite{serre-tate}).  If~$\ell$ is a
prime of $K$, let $D_{\ell}$ be a choice of decomposition group for
some prime~$\p$ of~$M$ lying over~$\ell$, and let $I_{\ell}$ be the
inertia group.  We haven't defined inertia and decomposition groups
for infinite Galois extensions, but the definitions are almost the
same: choose a prime of $\O_M$ over~$\ell$, and let $D_{\ell}$ be the
subgroup of $\Gal(M/K)$ that leaves~$\p$ invariant.  Then the
submodule $T_p(E)^{I_{\ell}}$ of inertia invariants is a module for
$D_{\ell}$ and the characteristic polynomial $F_{\ell}(x)$ of
$\Frob_{\ell}$ on $T_p(E)^{I_{\ell}}$ is well defined (since inertia
acts trivially).  Let $R_{\ell}(x)$ be the polynomial obtained by
reversing the coefficients of $F_{\ell}(x)$.  One can prove that
$R_{\ell}(x) \in \Z[x]$ and that $R_{\ell}(x)$, for $\ell\neq p$ does
not depend on the choice of~$p$.  Define $R_{\ell}(x)$ for $\ell=p$
using a different prime $q\neq p$, so the definition of $R_{\ell}(x)$
does not depend on the choice of~$p$.
\begin{definition}
The $L$-series of $E$ is
$$
 L(E,s) = \prod_{\ell} \frac{1}{R_\ell(\ell^{-s})}.
$$
\end{definition}

A prime~$\p$ of $\O_K$ is a prime of \emph{good reduction} for~$E$ if
there is an equation for $E$ such that $E \mod \p$ is an elliptic
curve over the field $\O_K/\p$. If $K=\Q$ and $\ell$ is a prime of
good reduction for~$E$, then one can show that that
$R_{\ell}(\ell^{-s}) = 1 - a_\ell \ell^{-s} + \ell^{1-2s},$
where
$
  a_{\ell} = \ell + 1 - \#\tilde{E}(\F_\ell)
$
and $\tilde{E}$ is the reduction of a local minimal
model for~$E$ modulo~$\ell$.  (There is a similar statement
for $K\neq \Q$.)

One can prove using fairly general techniques that the product
expression for $L(E,s)$ defines a holomorphic function in some right
half plane of~$\C$, i.e., the product converges for all~$s$ with
$\Re(s)>\alpha$, for some real number~$\alpha$.
\begin{conjecture}\label{conj:holo}
The function $L(E,s)$ extends to a holomorphic
function on all~$\C$.
\end{conjecture}
%TODO proven by corollary(Hecke)?

\subsection{Modularity of Elliptic Curves over $\Q$}
Fix an elliptic curve $E$ over~$\Q$.  In this section we will explain
what it means for $E$ to be modular, and note the connection with
Conjecture~\ref{conj:holo} from the previous section.

First, we give the general definition of modular form (of weight~$2$).
The complex {\em upper half plane} is
$
  \h  = \{z  \in \C : \Im(z) > 0\}.
$
A {\em cuspidal modular form} $f$ of level~$N$ (of weight~$2$) is a holomorphic
function
$
   f : \h \to \C
$
such that $\lim_{z\to i\infty} f(z) = 0$ and for every integer matrix
$\abcd{a}{b}{c}{d}$ with determinant~$1$ and $c\equiv 0 \pmod{N}$, we have
$$
  f\left( \frac{az + b}{cz + d} \right)
         = (cz+d)^{-2} f(z).
$$

For each prime number $\ell$ of good reduction, let $a_\ell = \ell+1 -
\#\tilde{E}(\Fell)$.  If $\ell$ is a prime of bad reduction let
$a_\ell = 0,1,-1$, depending on how singular the reduction~$\tilde{E}$
of~$E$ is over $\Fell$.  If $\tilde{E}$ has a cusp, then $a_\ell=0$,
and $a_\ell=1$ or $-1$ if $\tilde{E}$ has a node; in particular,
let $a_\ell=1$ if
and only if the tangents at the cusp are defined over~$\Fell$.

Extend the definition of the $a_\ell$ to $a_n$ for all positive
integers~$n$ as follows.  If $\gcd(n,m)=1$ let $a_{nm} = a_n \cdot
a_m$.  If $p^r$ is a power of a prime~$p$ of good reduction, let
$$
 a_{p^r} = a_{p^{r-1}}\cdot a_p \,\,-\,\, p \cdot a_{p^{r-2}}.
$$
If $p$ is a prime of bad reduction let $a_{p^r} = (a_p)^r$.

Attach to $E$ the function
$$
  f_E(z) = \sum_{n=1}^{\infty} a_n e^{2\pi i z}.
$$
It is an extremely deep theorem that $f_E(z)$ is actually
a cuspidal modular form, and not just some random function.


The following theorem is called the modularity theorem for elliptic
curves over~$\Q$.  Before it was proved it was known as the
Taniyama-Shimura-Weil conjecture.
\begin{theorem}[Wiles, Brueil, Conrad, Diamond, Taylor]
Every elliptic curve over $\Q$ is modular, i.e, the function
$f_E(z)$  is a cuspidal modular form.
\end{theorem}

\begin{corollary}[Hecke]
  If $E$ is an elliptic curve over~$\Q$, then the $L$-function
  $L(E,s)$ has an analytic continuous to the whole complex plane.
%and
%  satisfies a functional equation (symmetry) that relates $L(E,s)$ to
%  $L(E,2-s)$.
\end{corollary}
%TODO some explanation/background for this?



%%% Local Variables:
%%% mode: latex
%%% TeX-master: "ant"
%%% End: